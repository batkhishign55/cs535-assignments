\documentclass{article}

% Packages for setting up page margins
\usepackage[margin=1in]{geometry}

\usepackage{graphicx, setspace, amsmath, mathtools, amssymb, url, float, listings}
\setlength{\parskip}{2mm}
\graphicspath{ {./images/} }

% Title
\title{CS535 Design and Analysis of Algorithms - Assignment 5}
\author{Batkhishig Dulamsurankhor - A20543498}
\date{\today} % Use \date{} for no date

\begin{document}

\maketitle

\begin{enumerate}
  \item Even if the pivot selection is random, the time complexity will not necessarily change.
  It is still $O(n)$ because we are running the same algorithm with a random selection.

  With the original Median of Medians algorithm, we know that the pivot will be selected between 30 to 70 percentile when the batch size is 5.
  However, with random selection, the pivot quality can become worse compared to the original algorithm.
  Because in worst case scenario, there is a batch with really skewed values to the bottom or top.
  And it is kept on chosen by the random selection on each recursion, resulting in the final answer.
  The possiblity of this is unlikely but not 0.
  It will still perform better than completely random selection.

  \item Professor Piks' is different from the normal skip list by introducing blocks of $k$ items.
  After tossing a fair coin, each item isn't directly promoted, instead it becomes a candidate to represent the whole block for the next level.
  Only one of them is chosen as a representative.
  If a coin toss for each item in the block is tail and there is no possible candidates, the block will not have a representative in the next level.

  \begin{enumerate}
    \item Let's analyze the probability that all the blocks have a representative at the second level.
    We know that each coin toss is fair $1/2$.
    The chance for $k$ items to all hit tail and produce no candidates is $P_{no\_cand}=0.5^k$.
  
    This makes the possiblity that there is at least one canditate is chosen $P_{cand}=1-P_{no\_cand}=1-0.5^k$.
  
    Let's say we had total of $n$ items. So the number of blocks is $n/k$.
    So the probability that all the blocks have a representative at the second level is $P_{all\_cand}=P_{cand}^{n/k}=(1-0.5^k)^{n/k}$.

    \item Let's calculate the expected number of blocks that do have a representative at the second level.
    We already know the possiblity for a block to have a candidate is $P_{cand}=1-0.5^k$.

    Therefore, the expected number of blocks having a representative is $cnt_{cand}=n/k*P_{cand}=n/k*(1-0.5^k)$.
  \end{enumerate}
  
  \item

\end{enumerate}


\end{document}