\documentclass{article}

% Packages for setting up page margins
\usepackage[margin=1in]{geometry}

\usepackage{graphicx, setspace, amsmath, mathtools, amssymb, url, float, listings}
\setlength{\parskip}{2mm}
\graphicspath{ {./images/} }

% Title
\title{CS535 Design and Analysis of Algorithms - Assignment 5}
\author{Batkhishig Dulamsurankhor - A20543498}
\date{\today} % Use \date{} for no date

\begin{document}

\maketitle

\begin{enumerate}
  \item Even if the pivot selection is random, the time complexity will not necessarily change.
  It is still $O(n)$ because we are running the same algorithm with a random selection.

  With the original Median of Medians algorithm, we know that the pivot will be selected between 30 to 70 percentile when the batch size is 5.
  However, with random selection, the pivot quality can become worse compared to the original algorithm.
  Because in worst case scenario, there is a batch with really skewed values to the bottom or top.
  And it is kept on chosen by the random selection on each recursion, resulting in the final answer.
  The possiblity of this is unlikely but not 0.
  It will still perform better than completely random selection.

  \item Professor Piks' is different from the normal skip list by introducing blocks of $k$ items.
  After tossing a fair coin, each item isn't directly promoted, instead it becomes a candidate to represent the whole block for the next level.
  Only one of them is chosen as a representative.
  If a coin toss for each item in the block is tail and there is no possible candidates, the block will not have a representative in the next level.

  \begin{enumerate}
    \item Let's analyze the probability that all the blocks have a representative at the second level.
    We know that each coin toss is fair $1/2$.
    The chance for $k$ items to all hit tail and produce no candidates is $P_{no\_cand}=0.5^k$.
  
    This makes the possiblity that there is at least one canditate is chosen $P_{cand}=1-P_{no\_cand}=1-0.5^k$.
  
    Let's say we had total of $n$ items. So the number of blocks is $n/k$.
    So the probability that all the blocks have a representative at the second level is $P_{all\_cand}=P_{cand}^{n/k}=(1-0.5^k)^{n/k}$.

    \item Let's calculate the expected number of blocks that do have a representative at the second level.
    We already know the possiblity for a block to have a candidate is $P_{cand}=1-0.5^k$.

    Therefore, the expected number of blocks having a representative is $cnt_{cand}=n/k*P_{cand}=n/k*(1-0.5^k)$.
  \end{enumerate}

  \item We can convert the problem to a graph problem.
  Each course represents a vertix and we add edges between two courses that have overlapping schedule.
  Now our goal translates to finding the maximum cuts in our undirected graph $G=(V,E)$ into 3 subsets (3 classrooms) to minimize the clashes.
  We need to maximize the cut to make sure there is as less conflict as possible within each subset.
  This is similar to Randomized Min-Cut problem discussed in the class.

  The algorithm works as follows:

  \begin{itemize}
    \item Find all overlaps.
    We have to check courses pairwise, so it takes $O(V^2)$.
    \item Like the original algorithm, we pick each course and add it to one of the subsets uniformly at random.
    When we add a vertex to a subset, we also have to contract the edges.
    The contraction time is bounded by $O(E)$ and since we have $V$ courses, the total time is $O(V*E)$.
    We repeat this until we assign all edges to the subsets.
  \end{itemize}

  The total time complexity of the algorithm is $O(V^2+V*E)$.

  Let's calculate the expected number of non-overlapping pairs using this algorithm.
  The probability of any pair of courses that have overlap assigned to different classrooms is:

  $P_{no\_overlap}=1-P_{overlap}=1-1/3=2/3$.

  This is because we have 3 classrooms and the chances of two classes with the time conflict to be assigned into the same classroom is $P_{overlap}=1/3$.
  
  The expected number of non-overlapping is therefore $E*P_{no\_overlap}=2/3*E$.

  Since the optimal number of non-overlapping pairs $c^*$ cannot exceed $E$, the total number of conflicts:

  $E*P_{no\_overlap}=2/3*E\geq 2/3*c^*$.


  


\end{enumerate}


\end{document}