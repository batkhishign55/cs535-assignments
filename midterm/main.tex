\documentclass{article}

% Packages for setting up page margins
\usepackage[margin=1in]{geometry}

\usepackage{graphicx, setspace, amsmath, mathtools, amssymb, url, float}
\setlength{\parskip}{2mm}
\graphicspath{ {./images/} }

% Title
\title{CS535 Design and Analysis of Algorithms - Midterm}
\author{Batkhishig Dulamsurankhor - A20543498}
\date{\today} % Use \date{} for no date

\begin{document}

\maketitle

\begin{enumerate}
  \item
  \begin{enumerate}
    \item For subset system $(E,\mathcal{I})$ to be a matroid, it must satsify the following three properties.
    Here I am making assumption that $|E|>0$ or the network $G$ has at least one edge.

    \begin{itemize}
      \item First, it has to be a finite and non-empty set. It is trivial and subsets are limited by $k$ number of edges.
      If we assume that $|E|>0$ meaning we have at least an edge in the network, this property is satisfied.
      \item The second is heredity property.
      In this network $G$, if a subset $S_1$ of $E$ is an independent set, then $S_2$ that is a subset of $S_1$ is also an independent set.
      In other words, $S_2\subset S_1$ and if $S_1\in \mathcal{I}$ then $S_2\in \mathcal{I}$.
      Let's say $k=1$, so we can have at most one element in a subset.
      We have an independent set $|S_1|=k$, so a proper subset of $S_1$ is $\{\varnothing\}$ and by definition we know it also a part of independent sets.
      Therefore, it does satisfy the second property.
      \item The third property is exchange property.
      As long as $k>0$, we can always find an independent set $E_1$ which has more elements than an independent set $E_2$.
      So by exchange property, $E_1,E_2\in\mathcal{I}$ and if $|E_1|>|E_2|$ then we can always find an edge $e\in E_1$ such that $E_2\cup\{e\}\in \mathcal{I}$.
      So, it does satisfy the third property.

    \end{itemize}
    $(E,\mathcal{I})$ is a matroid assuming that the network $G$ has at least one edge.

    \item For a subset system $(E,\mathcal{J})$ to be a matroid, it has to satisfy the following three properties.

    \begin{itemize}
      \item First, it has to be a finite and non-empty set.
      We know $E$ has at least $e_1$ and $e_2$, so it is non-empty.
      With finite number of edges $E$, the first property is satisfied.
      \item The second is heredity property.
      Heredity property defines that a subset of independent set is always an independent set.
      Let's say an independent set $E_1$ contains $e_1$ and not $e_2$.
      Then the subset of $E_1$, that is $E_2$, can contain $e_1$ but never $e_2$. If $e_1\in E_1$, $e_2\notin E_1$ and $E_2\subseteq E_1$ then $e_2\notin E_2$.
      In other words, $e_1$ and $e_2$ cannot be in the same independent set.
      This doesn't break the rule defined in the problem that $\mathcal{J}$ contains at most one of $e_1$ and $e_2$ and satisfies the heredity property.
      \item The third property is exchange property.
      Exchange property defines that for $E_1,E_2\in\mathcal{J}$, if $|E_1|>|E_2|$ then $\exists e\in E_1-E_2$ such that $E_2\cup\{e\}\in\mathcal{J}.$
      Let's provide a counter example that doesn't satisfy this property.
      Suppose we have independent sets $E_1=\{e_1,e_3\}$ and $E_2=\{e_2\}$ where $e_2$ and $e_3$ are dependent on each other.
      This satisfies the rule defined in the problem that $\mathcal{J}$ contains at most one of $e_1$ and $e_2$.
      By exchange property, there should be some element in $E_1$ such that adding it makes $E_2$ independent set.
      We know that $e_2$ and $e_3$ are dependent on each other, making $e_1$ the answer.
      However, we can have at most one of $e_1$ and $e_2$ in $\mathcal{J}$ and to satisfy exchange property, we have to break this rule.
      For this reason, it doesn't satisfy exchange property.
    \end{itemize}
    Therefore, $(E,\mathcal{J})$ doesn't form a matroid.
  \end{enumerate}
  \item Since technical analyst Dr. W.Ho. Cares's algorithm doesn't terminate after finding MST, we have to check all the other edges left in the graph.
  Because we found MST, we have only one set in our union find data structure where all vertices are in.
  So finding an edge always results in a cycle, or both vertices have the same ancestor.
  Also, it is given that Dr. W.Ho. Cares is using path compression making the union find data structure more efficient.

  I think remaining $O(m)$ finds take only $O(m)$ steps is true.
  An actual cost for find is the height of vertex $n_i$, $h(n_i)$.
  We can define potential function as the complexity of the set.
  It reduces everytime find operation is executed, because the tree is compressed and becomes flatter.
  By path compression, all vertices from $n_i$ up to the root is has to be connected to the root except from the upmost vertices that is already connected to the root.
  So the change in potential function becomes $-(h(n_i)-1)$.

  The following is an amortized cost of a single find operation:

  $AM_i=Actual_i+\Delta PF_i=h(n_i)+(-h(n_i)-1)=h(n_i)-h(n_i)+1=1$.

  So the total amortized cost is:

  $\sum_{i}^{m} AM_i = \sum_{i}^{m} Actual_i+\sum_{i}^{m} \Delta PF_i=m*1=m$

  Therefore, the remaining $O(m)$ finds takes $O(m)$ so Dr. W.Ho Cares's claim is true.

  \item In a normal Binomial heap, the number of nodes in a tree of rank $r$ is $2^r$.
  If we allow up to $k$ trees of any given rank, we can have $k+1$ number of trees of rank $r-1$ for the of rank $r$.
  This is true for $k=1$, the normal Binomial heap because a tree of a rank $r$ has $k+1=2$ trees of rank $r-1$.
  Let's say $k=2$, then a tree of a rank $r$ has $k+1=3$ trees of rank $r-1$.
  We can see the pattern here, so the number of nodes a modified Binomial heap that can have up to $k$ trees of a given rank is $(k+1)^r$.

  \begin{figure}[H]
    \centering
    \includegraphics[width=0.5\textwidth]{image1.png}
    \begin{minipage}{0.5\textwidth}
        \centering
        Example: $k=2$, $r=2$ tree. We can count that the total number of nodes in this tree is $(k+1)^r=(2+1)^2=9.$
    \end{minipage}
  \end{figure}
  \item
\end{enumerate}


\end{document}